\PassOptionsToPackage{unicode=true}{hyperref} % options for packages loaded elsewhere
\PassOptionsToPackage{hyphens}{url}
%
\documentclass[english,]{book}
\usepackage{lmodern}
\usepackage{amssymb,amsmath}
\usepackage{ifxetex,ifluatex}
\usepackage{fixltx2e} % provides \textsubscript
\ifnum 0\ifxetex 1\fi\ifluatex 1\fi=0 % if pdftex
  \usepackage[T1]{fontenc}
  \usepackage[utf8]{inputenc}
  \usepackage{textcomp} % provides euro and other symbols
\else % if luatex or xelatex
  \usepackage{unicode-math}
  \defaultfontfeatures{Ligatures=TeX,Scale=MatchLowercase}
\fi
% use upquote if available, for straight quotes in verbatim environments
\IfFileExists{upquote.sty}{\usepackage{upquote}}{}
% use microtype if available
\IfFileExists{microtype.sty}{%
\usepackage[]{microtype}
\UseMicrotypeSet[protrusion]{basicmath} % disable protrusion for tt fonts
}{}
\IfFileExists{parskip.sty}{%
\usepackage{parskip}
}{% else
\setlength{\parindent}{0pt}
\setlength{\parskip}{6pt plus 2pt minus 1pt}
}
\usepackage{hyperref}
\hypersetup{
            pdftitle={Venetian Vostizza},
            pdfauthor={Jennifer Glaubius},
            pdfborder={0 0 0},
            breaklinks=true}
\urlstyle{same}  % don't use monospace font for urls
\usepackage{longtable,booktabs}
% Fix footnotes in tables (requires footnote package)
\IfFileExists{footnote.sty}{\usepackage{footnote}\makesavenoteenv{longtable}}{}
\usepackage{graphicx,grffile}
\makeatletter
\def\maxwidth{\ifdim\Gin@nat@width>\linewidth\linewidth\else\Gin@nat@width\fi}
\def\maxheight{\ifdim\Gin@nat@height>\textheight\textheight\else\Gin@nat@height\fi}
\makeatother
% Scale images if necessary, so that they will not overflow the page
% margins by default, and it is still possible to overwrite the defaults
% using explicit options in \includegraphics[width, height, ...]{}
\setkeys{Gin}{width=\maxwidth,height=\maxheight,keepaspectratio}
\setlength{\emergencystretch}{3em}  % prevent overfull lines
\providecommand{\tightlist}{%
  \setlength{\itemsep}{0pt}\setlength{\parskip}{0pt}}
\setcounter{secnumdepth}{5}
% Redefines (sub)paragraphs to behave more like sections
\ifx\paragraph\undefined\else
\let\oldparagraph\paragraph
\renewcommand{\paragraph}[1]{\oldparagraph{#1}\mbox{}}
\fi
\ifx\subparagraph\undefined\else
\let\oldsubparagraph\subparagraph
\renewcommand{\subparagraph}[1]{\oldsubparagraph{#1}\mbox{}}
\fi

% set default figure placement to htbp
\makeatletter
\def\fps@figure{htbp}
\makeatother

\ifnum 0\ifxetex 1\fi\ifluatex 1\fi=0 % if pdftex
  \usepackage[shorthands=off,main=english]{babel}
\else
  % load polyglossia as late as possible as it *could* call bidi if RTL lang (e.g. Hebrew or Arabic)
  \usepackage{polyglossia}
  \setmainlanguage[]{english}
\fi

\title{Venetian Vostizza}
\author{Jennifer Glaubius}
\date{}

\begin{document}
\maketitle

{
\setcounter{tocdepth}{1}
\tableofcontents
}
\hypertarget{preface}{%
\chapter*{Preface}\label{preface}}
\addcontentsline{toc}{chapter}{Preface}

\textbf{This book-in-progress was last updated on 21 May 2020}

\emph{Venetian Vostizza} is an examination of the Second Period of Venetian Domination
(1685 - 1715) in the territory of Vostizza in the northern Peloponnesus of Greece. This project began as my M.A.~thesis (Classics - 2005) at the University of Cincinnati, which was suggested by my advisor. Since then, I have revisited the material in a multivariate statistics course, where I analyzed the villas using contingency analysis, and in various presentations.

My analysis of Venetian Vostizza is unpublished except for the M.A.~thesis {[}TODO - add citation{]}. This project is intended both to examine the cadastral data using new methods for analysis and visualization and to expand knowledge of this period. My training since my M.A.~has been in Geography and it is with a geographic framework that I reexamine Venetian Vostizza. In particular, I bring a view that humans and the environment are intertwined and that each must be examined in light of the other.

The current form of this reanalysis of Venetian Vostizza as a bookdown web book with embedded leaflet.js maps was inspired by On The Line {[}TODO - add ref; \url{https://ontheline.trincoll.edu/}{]}, which helpfully includes how to create a similar project in Chapter 9 Mapping and Publishing \emph{On The Line}. Before finding this resource, I had been struggling with how to include all the analyses I envision for Venetian Vostizza onto a single web map. Now I can include useful maps as part of a larger narrative within a chapter and the book itself.

Up to this point, Venetian Vostizza has been a solo project, but I welcome collaboration. In the near future, I will make the cadastral database available as an API and make the geojson files available via Github. If you use the data and would like to contribute a chapter about your analysis, please contact me so we can collaborate.

Less formally, please add annotations and comments to \emph{Venetian Vostizza} using hypothes.is {[}TODO - add link; \url{https://web.hypothes.is/}{]}. To start using hypothes.is, go to the hypothes.is website and click Get Started. Then follow the steps of 1) creating a free account, 2) adding the Chrome extension or bookmarklet to your browser, and 3) annotating.

This project is still in its infancy. One of the issues I had grappled with is that I do not yet know the best way to tell the story of Venetian Vostizza. In this exploratory phase, I will make the chapters of my M.A.~thesis available here (please forgive the pie charts), as well as new writing, maps, and analysis of the topics I feel are most important: 1) how the cadastral records were collected; 2) how does the 2nd Period of Venetian Rule fit into the overall history of the region; 3) what were the patterns of population, including depopulation and immigration, during the 2nd Period of Venetian Rule; 4) what was the economy of Vostizza like before, during, and after the 2nd Period of Venetian Rule; 5) how much did the landscape and environment affect population and economic trends in Venetian Vostizza; 6) what was the built environment like in Venetian Vostizza; 7) how much do boundaries matter. These topics are subject to change as analysis and maps are created and described.

In the next phase of this work, the chapters will be reorganized to create a narrative for the history of Venetian Vostizza.

\hypertarget{introduction}{%
\chapter{Introduction to Venetian Vostizza}\label{introduction}}

TODO - add text

\hypertarget{history}{%
\section{Historical Context for Venetian Vostizza}\label{history}}

TODO - revise and expand text

Mainland Greece lost all political autonomy in 146 B.C. when the Romans crushed a revolt of the Achaean League and made Greece a dependency.24 From 146 B.C. until around the time of the Fourth Crusade in the early 13th century, Greece was mostly under Roman and later Byzantine control. From the 580s AD until the early 8th century, Roman rule was disrupted by attacks by Slavs, some of whom settled in Central Greece and the Peloponnese.25 The Normans had made previous attempts to occupy Greece in the late 12th century.26 After the attack and capture of Constantinople by the Crusaders in 1204, the Byzantine Empire was divided among the Crusaders and their allies.27 From this time, various Western European powers (including Franks, Flemings, and Italians, such as the Genoese, Florentines, and Venetians) had possessions in Greece (Fig. 3).28
The Venetians had already obtained influence within the Byzantine Empire by 1082, when the Byzantine Emperor gave them the right of free trade in exchange for help against the Normans.29 While much of the former Byzantine territory was doled out to various individuals, Venice obtained control of Durazzo, Corfu, Modon (modern Methoni), Coron and Cerigo (modern Kythera) by 1207, Euboea by 1209, and later began to take control of Crete in 1211.30 Venice held Modon, Coron, Euboea and Crete for over 200 years.31 After 1204, the Peloponnese was ruled by various nationalities: Franks, Flemings, Piedmontese, and finally the Navarrese Company in 1383.32 The resurgent Byzantine Empire conquered the Morea in 1432, but lost it to the Ottoman Turks by 1461.33 From 1377 on, various castles had been sold, given or leased to the Venetians in an attempt to keep them out of Greek or Turkish control.34 With the exception of Cyprus and the various Venetian holdings, there were no remaining Latin possessions in Greece.
After the First Veneto-Turkish war of 1463-79, in which Venice briefly held the Morea, the Venetians slowly lost possession of Greek territories to the Turks. By the 17th century all that remained was Crete, six of the seven Ionian islands, Tenos, Corfu, Butrinto, and Parga.35 The entire island of Crete, Venice's longest-standing possession in the Levant, was lost to the Ottoman Empire in the Cretan war (1645-69) except for the fortresses of Grabusa, Suda, and Spinalonga.36
In the late 17th century, the Ottoman Turks suffered military defeats and the Venetians gained a foothold in Greece once more. In 1683, the Turks were defeated at Vienna and effectively stopped from conquering more of Europe.37 To ensure that the Turks would not be able to continue their conquests in Europe, a ``Holy League'' was formed by the Holy Roman Emperor, Poland, and Venice with the blessing of Pope Innocent XI. Venice immediately declared war on the Turks and placed Francesco Morosini, who had defended Crete in the Cretan war, in charge of the army.38 The Ottoman Turks were distracted from Greece at this time, because they were also at war with Russia.39
In 1685, Morosini began military operations in the southern Peloponnese with an army comprised of mercenaries from various European nations.40 During 1685, the Venetians gained Coron, Zarnata, Kelepha, and Kalamata in the southern Peloponnese. The rest of the southern Peloponnese and Argolid, with the exception of Monemvasia, was conquered during the 1686 campaign season. The entire Morea, except Monemvasia, which was not surrendered by the Turks until 1690, and Athens were under Venetian control.41 After a failed attempt to capture Euboea, in the winter of 1687, Morosini decided to abandon Athens and relocate the Greek inhabitants to other Venetian- controlled areas.42 The Athenians left the city on March 24, 1688 for the island of Zante, various parts of the Morea, or - the preferred option for the Athenians - nearby Salamis.43 In 1699, the Treaty of Karlowitz formally ended hostilities and gave possession of the Morea to the Venetians.44
The Ottoman Empire declared war on Venice in 1714, after Russia had been defeated. By October, 1715, the entire Morea and the three Venetian bases on Crete had been lost to the Turks. The Morea was quickly conquered by the Ottoman Empire in part because most fortifications were surrendered by the Venetian commanders, with little or no resistance.49 TODO: explore whether Greek hatred really led to rapid defeat. One reason for the rapid advancement of the Ottoman Turks was the Greek hatred of Venice's policies. The problems between the Latin Catholic church, which the Venetians insisted have authority over the Morea, and the Orthodox Patriarch, oppression of any industries that would be in competition with Venice, and the quartering of soldiers in Greek households, made the Greek inhabitants of the Morea eager for the return of the Turks.50 Peace was negotiated between the Turks and Venetians in 1718 by the treaty of Passarowitz, which left Venice only with Kythera, Antikythera, Butrint (Albania), Lefkas, Prevesa and Vonitza.51 The Morea remained under Ottoman domination until the 19th century when the Greek War of Independence brought political autonomy to Greece.

24 CAH3 v. IX, p.~32 states that L. Mummius put down a revolt of the Achaean League in 146B.C. Afterwards the cities in Greece paid tribute to Rome although Greece did not receive a governor until the late 1st century B.C. (Heurtley 1965, p.~28).
25 CAH2 v.14, 727; Darby 1965, pp.~41-48. Knowledge of Slavic settlement in the Peloponnese comes from the Chronicle of Monemvasia. For the veracity of this document see Charanis 1950. The three versions of the text can be found in Bees 1979.
26 Darby 1965, p.~51. Failed Norman conquests of Greece include 1081-1084; 1106; 1146; 1185.
27 Whether the Venetians persuaded the Crusaders to attack Constantinople or the Crusaders had it in mind already is not entirely known. Darby (1965, pp.~52-53) is of the opinion that the Venetians pursuaded the Crusaders to divert first to Zara and then to Constantinople in order to pay off their passage to the Levant. McNeill (1974, p.~30) explains that the diversion away from the Holy Land may have had more to do with the personal interests of the commander-in-chief of the Crusaders, Boniface of Montferrat. Lock (1995, p.~141) notes that the capture of Constantinople was caused by instability in the city, rather than direct Venetian influence. For more information about the Latin states in Greece after the Fourth Crusade see: Darby 1965, pp.~52-72; Miller 1908; Lock 1995.
28 Topping 1949; 1956; 1965; 1966a; 1966b; 1976b; Bon 1969.
29 Darby 1965, p.~51; McNeill 1974, p.~3. The trade agreement between the Venetians and the Byzantine Empire was not stable. The amount of concessions given to the Venetians varied over time and other Italian cities, such as Genoa, were also given certain priviledges. See McNeill 1974, p.~37; Lock 1995, pp.~137- 140.
30 McNeill 1974, p.32; Darby 1965, pp.~66-68.
31 A collection of Venetian documents about Coroni and Methoni (Nanetti 1999) vividly illustrates Venetian rule in their colonies. Crete was held by the Venetians for over 400 years.
32 Darby 1965, pp.~61-63; Lock 1995; Miller 1908; Jacoby 1989; Angold 1989; Luttrell 1989; Balard 1989. 33 Darby 1965, pp.~62-63.
34 Lock 1995, p.~160. Nauplia and Argos in 1388; Naupaktos in 1407; Patras leased in 1408; Navarino in 1423; Monemvasia in 1463.
35 Darby 1965, p.~82; Miller 1921, p.~403; Slot 1982, pp.~82-116, 176-177.
36 Darby 1965, pp.~83-84; Lock 1995, p.~160; Slot 1982, pp.~162-192.
37 Miller 1921, p.~403.
38 Miller 1921, p.~403; Mackenzie 1992, p.~18.
39 Miller 1921, p.~406.
40 Miller 1921, p.~404; Mackenzie 1992, p.~18.
41 About the conquest of the Morea: Miller 1921, pp.~404-409, 414; Mackenzie 1992, p.~18. The Venetian capture of Athens: Miller 1921, pp.~406-409; Mackenzie 1992, pp.~18-19; Paton 1940. It was at this time that the Parthenon was destroyed by the Venetian bombardment of Turkish fortifications on the Acropolis. 42 The decision to abandon Athens was affected by an outbreak of the plague in Greece (Miller 1921, pp.~412-413; Mackenzie 1992, pp.~20-21). The Athenians were opposed to leaving their property in Attica, but were promised land and other compensation when they reached the Morea (Miller 1921, p.~414; Mackenzie 1992, p.~21). When Morosini left Athens, he sent four lions scavenged from Piraeus to adorn the Arsenal in Venice (Miller 1921, p.~413; Mackenzie 1992, p.~22).
43 Miller 1921, pp.~413-414; Mackenzie 1992, p.~21.
44 Darby 1965, p.~84; Miller 1921, p.~417
49 Miller 1921, pp.~424-426. For an account from the Ottoman side, see Brue 1870. 50 Darby 1965, p.~84; McNeill 1974, pp.~220-221.
51 Darby 1965, p.~85; Miller 1921, p.~426.

\hypertarget{sources}{%
\section{Sources of Data on Venetian Vostizza}\label{sources}}

TODO: add C.O. and C.P.

TODO: add territory map in Vienna Kriegsarchiv

TODO: add census in Panagiotopolous

TODO: add traveler's accounts

\hypertarget{definitions}{%
\section{Definitions}\label{definitions}}

TODO: add text

\hypertarget{cadasters}{%
\chapter{Venetian Cadasters for Vostizza}\label{cadasters}}

TODO: revise and expand text

In order to identify which land was deserted and also to create a system of taxation based on ownership of land, the Venetians immediately began a program of land-survey in 1687. Three directors were appointed to take charge of the survey of the Morea (sindici e catasticatori): Marin Michiel, Gierolimo Renier, and Domenico Gritti.\footnote{Peter Topping, \emph{Co-Existence of Greeks and Latins in Frankish Morea and Venetian Crete} (na, 1976), p.~310; Peter Topping, ``PREMODERN PELOPONNESUS: THE LAND AND THE PEOPLE UNDER VENETIAN RULE (1685‐1715)* Premodern Peloponnesus,'' \emph{Annals of the New York Academy of Sciences} 268, no. 1 (1976): 92--108, \url{https://doi.org/10/cjthq5}, pp.~92-93. Renier died during the first year of work, leaving the burden of the land survey to fall upon Michiel and Gritti. The more detailed Catastico Particolare for Vostizza was accomplished by Francesco Grimani.} The catastico ordinario, an account of each settlement in a territory, was completed for almost the entire Peloponnese by Michiel and Gritti. The catastico particolare, a listing of plots of land in each luogo, was only completed for the territories of Elis, Fanari, Karitaina, Argos, Vostizza, and Romania.\footnote{Topping, p.~96. The cadaster for Nauplion is discussed in Eftyhia Liata, ``Το Ναύπλιο και η ενδοχώρα του από τον 17 ο στον 18 ο αιώνα. Οικιστικά μεγέθη και κατανομή της γης,'' \emph{Οικιστικά μεγέθη και κατανομή της γης. Athens: Academy of Athens}, 2002.} The two land-surveys of Vostizza were submitted to the Venetian Senate in 1700.

\hypertarget{landscape}{%
\chapter{Landscape of Vostizza}\label{landscape}}

TODO: revise text and add more info

The Venetian territory of Vostizza is located in the northern Peloponnese along the Gulf of Corinth (Fig. 1). The total land area of the Vostizza region according to Venetian surveyors is 149 square km (152,136 stremma and 349 tavole).\footnote{Konstantinos. Dokos and Georgios D. Panagopoulos, \emph{To Venetiko ktematologio tes Vostitsas} (Athena: Morphotiko Institouto Agrotikes Trapezas, 1993), p.~6. This conversion to km2 is made under the assumption that the stremma consisted of 324 tavole, with a tavole equalling 3.02 m2. A Venetian tavole equalled 25 sq. piede (feet). A Venetian piede was 0.348 m {\textbf{??}}, pp.~xxx; 136, 283). See Siriol Davies, ``Pylos Regional Archaeological Project, Part VI: Administration and Settlement in Venetian Navarino,'' \emph{Hesperia}, 2004, 59--120, \url{https://doi.org/10/d7n9kq}, appendix 2, pp.~113-116. The modern area of Aegiale, which is smaller than the Venetian territory of Vostizza, is 103 square kilometers in area (\textbf{TODO: fix ref} Ministère de L'Économie Nationale 1914, table D, p.~vi.).} The territory is bounded by the Gulf of Corinth to the north. The northern shore of the Gulf of Corinth is visible from most of the shore in the Vostizza territory.

The primary city of the region is Aigion, named Vostizza Terra (town) in Venetian records. The perceived potential of Aigion as a harbor varies by source. According to the late 17th century traveler to the Morea, Bernard Randolph, the town of Aigion (Vostizza) did not have a port.\footnote{Bernard Randolph, \emph{The Present State of the Morea, Called Anciently Peloponnesus: Together with a Description of the City of Athens, Islands of Zant, Strafades, and Serigo. With the Maps of Morea and Greece, and Several Cities. Also a True Prospect of the Grand Serraglio, or Imperial Palace of Constantinople, as It Appears from Galata: Curiously Engraved on Copper Plates}, vol. 18 (W. Notts, 1966), p.~3.} Sir William Gell, writing in 1817, reported that ``the anchorage is not safe with a northerly wind,'' which implies that the town did have a dependable harbor.\footnote{{\textbf{??}}, p.~7.} Colonel Leake in 1830 noted that the harbor at Aigion (Vostitza) was safer than the port at Patras, although it was very deep near the shore.\footnote{{\textbf{??}}, pp.~182-183.}

The terrain in the territory of Vostizza is mostly mountainous with a small coastal plain around the town of Aigion. The elevation of the land ranges from sea level to 1770 m above sea level.\footnote{Dokos and Panagopoulos, \emph{To Venetiko ktematologio tes Vostitsas}, pp.~698-699.} The plain of Aigion, between the Erineos and Selinous rivers, is wide and is cultivated today with vines and fruit trees (Fig. 2). To the east of Aigion, the mountains crowd the coastline, leaving only 400 yards (366 m) of plain between them and the sea. The primary mountains in the area are part of the Aroania Mountains and Mt. Marmati. The rivers of the region include, in addition to the Erineos and Selinous, the Vouraikos, Krathis, and Krios.

The climate in Vostizza is typically Mediterranean with hot, dry summers and wet, cooler winters. Temperatures for the Peloponnese can range from 0° C in the winter to 48° C in the summer. The majority of rainfall accumulates during the winter months of November, December, and January.\textbf{TODO: update climate data and add refs} The vegetation of Vostizza, as determined by the climate, belongs to the True Mediterranean Region, southern zone. The foothills and mountain regions today are covered mainly with evergreen woods and maquis. Aleppo and stone pines are the most characteristic types of trees for the area.\textbf{TODO: update veg data and add refs} The land was more arid and had fewer trees than today in the early 19th century.\footnote{Alfred Thomas Grove and Oliver Rackham, \emph{The Nature of Mediterranean Europe: An Ecological History} (Yale University Press, 2003), p.~185.} The inventory of trees in the catastico ordinario for Vostizza shows that olives, grapes, mulberries, oranges, apples, quinces, figs, peaches, pears, plums, pomegranates, lemons, cherries, and almonds were all grown in the area in 1700. Since that time currants have become widely cultivated on the plain of Aigion.

\hypertarget{present-landscape}{%
\section{Present-day Landscape}\label{present-landscape}}

TODO: add text

\hypertarget{venetian-landscape}{%
\section{Landscape During the Venetian Era (1685-1715)}\label{venetian-landscape}}

TODO: add text

\hypertarget{sites}{%
\chapter{Sites in Venetian Vostizza}\label{sites}}

TODO: add text

\hypertarget{population}{%
\chapter{Population of Venetian Vostizza}\label{population}}

\textbf{TODO: revise and expand text}

The Athenians were not the only people who emigrated to the Morea during the Venetian occupation. When the Venetians began taking stock of their new territory, they found that it was severely depopulated and underutilized. To solve these problems, the Venetians brought settlers from the northern coast of the Corinthian Gulf, Chios, and the Ionian islands into the Peloponnese. Among the newcomers to the Morea were 636 families from Salona who settled in the Vostizza region.\footnote{Topping, \emph{Co-Existence of Greeks and Latins in Frankish Morea and Venetian Crete}, p.~314, 322; Topping, ``PREMODERN PELOPONNESUS.'', p.~93; Kōnstantinos D̲okos, ``Hē Sterea Hellas Kata Ton Henetotourkikon Polemon, 1684-1699 Kai Ho Salōnōn Philotheos'' (Hetaireia Stereoelladikōn Meletōn, 1975) 1975.} The overall program of resettlement was successful and the population of the Morea rose from a 1692 calculation of 116,000 to over 250,000 by 1708.\footnote{{\textbf{??}}, p.~84}

\hypertarget{economy}{%
\chapter{Economy of Venetian Vostizza}\label{economy}}

TODO: add text

\#Taxation in Venetian Vostizza \{\#taxes\}

TODO: add text

\#Immigration in Venetian Vostizza \{\#immigration\}

TODO: add text

\hypertarget{mapping}{%
\chapter{Mapping and Analysis}\label{mapping}}

\hypertarget{phase-1-exploratory-data-analysis}{%
\section{Phase 1: Exploratory Data Analysis}\label{phase-1-exploratory-data-analysis}}

Originally, all mapping for this project was done using GIS, namely ArcGIS. In the first stages, exploratory data analysis was completed after data from the Venetian cadasters had been combined with their spatial locations digitized from the map of the territory. However, these spatial data could not be linked to other spatial data since the map was not properly georeferenced to real world coordinates, which was difficult owing to problems with matching up the map from 1700 with present-day locations. Most maps that were created were choropleth or included pie charts for each villa. The data at this point was within a Filemaker database with relational tables to connect data.

\hypertarget{mapping-1}{%
\subsection{Mapping}\label{mapping-1}}

\textbf{TODO: revise section from thesis below}
\textbf{TODO: incorporate footnotes}
Both modern maps and a Venetian map of Vostizza dating to 1700 were used to create the layers in the GIS. Five map sheets that contain the territory of Vostizza\footnote{These maps were created by the Hellenic Military Geographical Service. All are in the 1:50,000 m series. Aigion (022), Amygdalea (035), Chalandritsa (377), Dervenion (089), and Navpaktos (233).} were combined into a single map in Photoshop, and then added to the GIS. The latitude and longitude of the composite modern map was registered with the GIS program to facilitate adding modern data in the future.
Next, the Venetian map of the territory of Vostizza in 1700\footnote{This map is located in the \textbf{TODO add ref/link to map in archive} Kriegsarchiv of the Austrian State Archives in Vienna (B III a 121); see Olga Katsiarde-Hering, \emph{Venezianische Karten Als Grundlage Der Historischen Geographie Des Griechischen Siedlungsraumes (Ende 17. Und 18. Jh.)}, 1993, p.~301.} was scanned and added to the GIS. Once added, the 1700 map was put into proper spatial position through a process called ``rubber sheeting.'' In rubber sheeting, a map with an unknown coordinate system, such as the 1700 map, is placed on a map with known coordinates, in this case the modern map, by connecting the same points on each map. Once this process was accomplished for the Venetian map of Vostizza, the territory was approximately in the correct position (Fig. 6). As is clearly shown in Fig. 6, the Venetian map does not entirely match the modern map. This incongruity between the modern map and 1700 map is a result of mapmaking techniques of the early 18th century.\footnote{For more information about the spatial relationship between Venetian maps ca. 1700 and modern maps see M. Wagstaff and S. Chrysochoou-Stavridou, ``Two Unpublished Maps of the Morea from the Second Venetian Period,'' in \emph{Proceedings V Conference of Peloponnesian Studies}, vol. 4, 1996, 289--316.}
After the Venetian map of Vostizza was positioned properly, new layers were created by digitizing features from the map. The boundaries of the territory and each villa, as well as the positions of rivers and settlements were all digitized using a method called ``heads up'' or ``on the fly'' digitizing. The two names for this type of digitizing refer to the fact that the process takes place on the computer screen, rather than using a digitizing tablet. Four layers were created using heads up digitizing. Two polygon layers, one only of the territorial boundaries of Vostizza (Fig. 7) and the other of the boundaries of the luoghi (Fig. 8) were made.68 The rivers were digitized into a polyline layer (Fig. 9) and the settlements were placed in a point layer (Fig. 10). This completed the spatial additions to the GIS project.

\hypertarget{database}{%
\subsection{Database}\label{database}}

The basic data about Vostizza in 1700 are recorded in the catastico ordinario and catastico particolare for Vostizza.69 Three databases were created from the general and detailed cadasters. Information from the catastico ordinario about each of the 34 luoghi in the territory was placed into a database named ``Luoghi.'' The data in the database ``Luoghi'' include information about number of houses, number of fruit and nut plants, and animals, and amounts of land under cultivation for each luogo (Fig. 11; Appendix 1 includes a description of all fields in the ``Luoghi'' database).
A second database for each separate agricultural field called ``Property,'' records the owner, type of ownership (public, beneprobatum, concessione)70, and amount of land (Fig. 12; Appendix 2 includes a description of all fields in the ``Property'' database). The Venetians recorded the land area with more than one measurement system. Vineyards are recorded by zappade, stremma, Campi Trevisan, and Campi Paduan.71 Other fields (terreni) omit measurements in zappade. Each measurement has been converted into square meters in the database.72
The third database -- ``Settlements'' -- corresponds to the point layer Settlements. The information in this database includes the name and designation of each settlement from the Venetian map of Vostizza, catastico ordinario, catastico particolare, and the Grimani census (Fig. 13). This database is used to illustrate the differences among the four sources in designations for settlements.

69 Dokos and Panagopoulos 1993; A.S.V. Grimani b.81/17-33 and b.81/56-279.
70 The categories of Venetian property ownership are defined in Chapter 5. The reader is referred to the Glossary for a short definition of the various Venetian terms used in this thesis.
71 The zappade is an areal unit of measurement for vineyards used in Greece. Stremmata are also Greek units of measurement for areas of land. The Trevisan and Paduan systems of areal measurement both used basic units called tavola, but of different sizes. See Zupko 1981, for differences in measurement between cities in Italy. In practice, the Trevisan and Paduan systems were often modified according to circumstance. See Topping 1972, esp.~p.~78, Forbes 2000, and Davies 2004 for interpretations of Venetian land measurement in the Morea.
72 The size of the stremma employed in the Peloponnese during the Ottoman and Venetian periods has been a matter of controversy. See Davies (2004, pp.~113-116, esp.~p.~114), who suggests that it was equivalent to 1889.37 m2 (for 625 tavole). Calculations based on the Vostizza cadastre, examined in this thesis, confirm that Venetian administrators imagined it to be equal to 1889.37 m2.

\hypertarget{phase-2-statistical-analysis}{%
\section{Phase 2: Statistical Analysis}\label{phase-2-statistical-analysis}}

The difficulty with georeferencing the Venetian map to real-world coordinates was overcome in the next phase of the mapping using Erdas Imagine. Using trial and error of matching locations on the coastline and settlement locations in the interior with present-day villages, a reasonable georectification was achieved. From this georeferencing, the villas, rivers, and settlements were again digitized. Although other sources of spatial data could now be added to the analysis, no further analysis was completed along this vein. Instead, better choropleth maps (without any pie charts) were created that displayed the results of the contingency table analysis. The Filemaker database tables were exported to csv format and often opened in Excel. The tables in this format were also joined to the appropriate spatial data layers in ArcGIS and later QGIS.

\hypertarget{phase-3-advanced-analysis-and-interactive-mapping}{%
\section{Phase 3: Advanced Analysis and Interactive Mapping}\label{phase-3-advanced-analysis-and-interactive-mapping}}

The current phase of mapping and analysis is the most ambitious yet. Planned analyses include least-cost path analysis between locations to analyze travel between villas, clustering analysis on the villas to look for patterns, and creation of interactive maps and story maps using leaflet.js.
TODO: add info about getting data from database into MongoDB using Mongoose

\hypertarget{resources}{%
\chapter{Helpful Resources for this project}\label{resources}}

TODO: add text

\hypertarget{refs}{}
\leavevmode\hypertarget{ref-daviesPylosRegionalArchaeological2004}{}%
Davies, Siriol. ``Pylos Regional Archaeological Project, Part VI: Administration and Settlement in Venetian Navarino.'' \emph{Hesperia}, 2004, 59--120. \url{https://doi.org/10/d7n9kq}.

\leavevmode\hypertarget{ref-dokosVenetikoKtematologioTes1993}{}%
Dokos, Konstantinos., and Georgios D. Panagopoulos. \emph{To Venetiko ktematologio tes Vostitsas}. Athena: Morphotiko Institouto Agrotikes Trapezas, 1993.

\leavevmode\hypertarget{ref-dokosHeStereaHellas1975}{}%
D̲okos, Kōnstantinos. ``Hē Sterea Hellas Kata Ton Henetotourkikon Polemon, 1684-1699 Kai Ho Salōnōn Philotheos.'' Hetaireia Stereoelladikōn Meletōn, 1975.

\leavevmode\hypertarget{ref-groveNatureMediterraneanEurope2003}{}%
Grove, Alfred Thomas, and Oliver Rackham. \emph{The Nature of Mediterranean Europe: An Ecological History}. Yale University Press, 2003.

\leavevmode\hypertarget{ref-katsiarde-heringVenezianischeKartenAls1993}{}%
Katsiarde-Hering, Olga. \emph{Venezianische Karten Als Grundlage Der Historischen Geographie Des Griechischen Siedlungsraumes (Ende 17. Und 18. Jh.)}, 1993.

\leavevmode\hypertarget{ref-liataNayplioKaiEndohora2002}{}%
Liata, Eftyhia. ``Το Ναύπλιο και η ενδοχώρα του από τον 17 ο στον 18 ο αιώνα. Οικιστικά μεγέθη και κατανομή της γης.'' \emph{Οικιστικά μεγέθη και κατανομή της γης. Athens: Academy of Athens}, 2002.

\leavevmode\hypertarget{ref-randolphPresentStateMorea1966}{}%
Randolph, Bernard. \emph{The Present State of the Morea, Called Anciently Peloponnesus: Together with a Description of the City of Athens, Islands of Zant, Strafades, and Serigo. With the Maps of Morea and Greece, and Several Cities. Also a True Prospect of the Grand Serraglio, or Imperial Palace of Constantinople, as It Appears from Galata: Curiously Engraved on Copper Plates}. Vol. 18. W. Notts, 1966.

\leavevmode\hypertarget{ref-toppingCoexistenceGreeksLatins1976}{}%
Topping, Peter. \emph{Co-Existence of Greeks and Latins in Frankish Morea and Venetian Crete}. na, 1976.

\leavevmode\hypertarget{ref-toppingPREMODERNPELOPONNESUSLAND1976}{}%
---------. ``PREMODERN PELOPONNESUS: THE LAND AND THE PEOPLE UNDER VENETIAN RULE (1685‐1715)* Premodern Peloponnesus.'' \emph{Annals of the New York Academy of Sciences} 268, no. 1 (1976): 92--108. \url{https://doi.org/10/cjthq5}.

\leavevmode\hypertarget{ref-wagstaffTwoUnpublishedMaps1996}{}%
Wagstaff, M., and S. Chrysochoou-Stavridou. ``Two Unpublished Maps of the Morea from the Second Venetian Period.'' In \emph{Proceedings V Conference of Peloponnesian Studies}, 4:289--316, 1996.

\end{document}
